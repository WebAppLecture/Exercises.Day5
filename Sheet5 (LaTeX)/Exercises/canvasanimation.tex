\Exercise{Eine einfache Canvas Animation}
%
\par Entwerfen Sie eine HTML Seite, die ein \htag{canvas}-Element sowie drei Buttons (Langsamer, Schneller, Abspielen / Pause) beinhaltet. Über den Button Abspielen soll eine Animation (in Dauerschleife) gestartet werden. Beim erneuten Klick auf den Button soll die Animation angehalten werden.
%
\par Gehen Sie am besten folgendermaßen vor:
%
\begin{itemize}
\item Erstellen Sie eine \jfunc{drawCanvas}-Methode, die mit dem entsprechenden Kontext Zeichnungen durchführt.
\item Beim Klicken auf den Startbutton soll über \jfunc{setInterval} ein gepulster Timer erstellt werden, der in regelmäßigen Abständen die \jfunc{drawCanvas}-Methode ausführt.
\item Beim erneuten Klick auf den Startbutton soll der Timer über \jfunc{clearInterval} gelöscht werden.
\item Schneller und Langsamer sollen nicht über die Pulsrate (d.h. Framerate).kontrolliert werden, sondern über Eigenschaften die in der \jfunc{drawCanvas}-Methode verwendet werden (fett markiert).
\end{itemize}
%
\begin{figure}[!h]
\centering
\includegraphics{Figures/car.png}
\caption{Die zu zeichnenden Objekte}
\label{fig:car}
\end{figure}
%
\par Folgendes soll dabei gezeichnet werden:
%
\begin{itemize}
\item Zwei Rechtecke (Orange und Grün)
\item Zwei Kreise (Blau)
\item Vier Linien in den Kreisen
\end{itemize}
%
\par Folgende Eigenschaften sollen animiert werden:
%
\begin{itemize}
\item Die Position des orangen Rechtecks, der Kreise und der Linien (soll eine Fahrt darstellen) über \jvar{dx}
\item Der Winkel der Linien relativ zum Untergrund (soll bewegte Räder darstellen) über \jvar{dalpha}
\end{itemize}
%
\par Sie können dies alles über Transformationen an den entsprechenden Stellen erreichen. Für die Position sollten Sie \jfunc{translate} verwenden, für die bewegten Räder \jfunc{rotate}. Vergessen Sie nicht die Transformationen entsprechend zurückzusetzen.
