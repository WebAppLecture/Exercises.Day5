\RequiredExercise{Transformationen und Übergänge}
%
\par Erstellen Sie eine Seite welche über fünf verschiedene \htag{div}-Boxen
verfügt. Jede Box soll das selbe Grundlayout besitzen, welches aus einem
Rahmen, einer Hintergrundfarbe und einer bestimmten Größe besteht. Außerdem
sollen die Boxen einen festen Abstand voneinander besitzen.
%
\par Stellen Sie für alle Eigenschaften der Boxen ein Übergangsszenario von
zwei Sekunden Dauer ein. Jede Box soll außerdem beim Drüberfahren mit der Maus
(\emph{hover}) anders transformiert werden:
%
\begin{itemize}
\item
Die erste Box erhält eine Translation von \qty{50}{px} (horizontal) und
\qty{-20}{px} (vertikal).
\item
Die zweite Box erhält eine Translation um \qty{-15}{px}, \qty{30}{px} und eine
Rotation um \qty{90}{\degree}.
\item
Die dritte Box erhält eine Rotation um \qty{45}{\degree}.
\item
Die vierte Box erhält eine Verzerrung um \qty{50}{\degree} (vertikal) und eine
Rotation um \qty{50}{\degree}.
\item
Die fünfte Box erhält eine Verzerrung um \qty{25}{\degree} (horizontal), eine
Translation um \qty{50}{px} (vertikal) und eine Rotation um \qty{25}{\degree}.
\end{itemize}
%
\par Spielen Sie zum Abschluss noch mit einer zusätzlichen Transformation des
Containers der Boxen herum (vermutlich ist das in Ihrem Fall das
\htag{body}-Element). Was können Sie hierbei feststellen?